\documentclass{book} 

\usepackage{graphicx}
\usepackage{amsmath}
\usepackage{algorithm}
\usepackage{algpseudocode}
\usepackage{float}
\usepackage{hyperref}
\usepackage{placeins}
\usepackage{amssymb}
\usepackage{listings}

\usepackage{tikz}
\usetikzlibrary{arrows}
\usetikzlibrary{snakes}
\usetikzlibrary{decorations.pathmorphing}

\hypersetup{
	colorlinks,
	citecolor=black,
	filecolor=black,
	linkcolor=black,
	urlcolor=black
}

\title{Online Contests Solutions}
\author{Saman Saadi}
\date{} 

\begin{document}
	\frontmatter
	\maketitle
%	\newpage
	\tableofcontents
	\mainmatter
	\chapter{HackerRank}
	\section{New Year Chaos}
	 You can find the question in this \href{https://www.hackerrank.com/challenges/new-year-chaos/problem?h_l=interview&playlist_slugs%5B%5D=interview-preparation-kit&playlist_slugs%5B%5D=arrays}{link}.
	 \par We define $index_i$ as the current index for person $i$. For example if we have $1, 2, 3, 4$ and $4$ bribes $3$, the queue looks like $1, 2, 4, 3$. So $index_4 = 3$. Since no body can bribe more than 2 times, $index_i \ge i - 2$ for $1 \le i \le n$. Consider person $n$. No body can bribe that person. So $n - 2 \le index_n \le n$. After we retruned that person to his actual place we can consider $n - 1$. So we have $n - 3 \le index_{n - 1} \le n - 1$ (note that at this moment $index_n = n$).
	 \begin{lstlisting}
void minimumBribes(vector<int> q) {

    const auto& n = q.size();
    int res = 0;
    for (int num = n; num > 0; --num)
    {
        for (int i = max(0, num - 3); i < num - 1; ++i)
        {
            if (q[i] == num)
            {
                ++res;
                swap(q[i], q[i + 1]);
            }
        }
        if (q[num - 1] != num)
        {
            cout << "Too chaotic" << endl;
            return;
        }
    }
    cout << res << endl;
}
	 \end{lstlisting}
\end{document}
